THE CODEX WITHIN
This is a tree core. See its rings? Every year a tree lives, it grows another ring.

But not all tree rings are created equal! Environmental events like forest fires or droughts can affect the growth of trees, which, in turn, can cause 'abnormal' tree ring formations. But what is abnormal, anyway?

Differences in one tree's rings speak to that tree's particular growth patterns over time. And since trees don't just uproot and leave, they must compete with other local flora for sun, water, and mineral resources. Do you see what happens?

So how do you determine the average growth patterns of a specific species within a particular geography?

Trees don't move much after taking root. So let's consider 'abnormal' as "distinguishable deviations from a tree's annual growth patterns over the course of a tree's lifetime as measured against proximate species' growth patterns."

Mapping differences between trees of the same species within a given region can speak to that species' diversity. The fancy word, "Dendrochronology," describes this study and the dating of tree rings over time.
There's another fancy word out there, "Phylogenetics," which describes the examination of evolutionary relationships among groups of organisms through lenses of molecular data sequencing and morphological data matrices.

THE CODEX WITHOUT
****New words**** often get uttered into existence by chance and circumstance. And out of these verbal mintings can come novel ways of communicating some-thing, some type of way. Maybe this :: that.
On Neologizing Some Type of Way
Out of the deepness, hurdling indirectly slantways, spiraled some strange slangers of wyrd:
Inquisiktr: "What's in a thing?"
Risplyndner: "Thing-ness!"
Inquisiktr: "Then, what makes up its thing-ness's '-ness'?"
Risplyndner: "'ness-itude!"
Inquisiktr: "Then why is it that, when you put one's personal type of thing-ness next to another person's way of thing-ness, that some-type-of-way gets made out of thing-ness?"
Risplyndner: "..."
Inquisiktr: "Perhappenchaybe."
Some Type of Way






BLAH-BLAH-BLAHS
But so few of these novel blah-blahs-blahs stick around, and even fewer get adopted at large. So, what determines the mimetic stickiness of new words?
Let's concentrate on words that get minted into language but fail to gain enough attentional mass to stick around.
Neologisms make new meaning by either remixing existing content within a new context, or by straight up inventing new word meanings - e.g. "xerox".
To learn how each of these rhetorical 'moves' really works, let's try to make our own.
Since puns employ etymology with elasticity - in responsive albeit variable ways - let's call puns "etymolastical."
Boom! New word!
And since neologisms typically initially occur once and in-situ, let's call them as "newonced."
Blaaaammm-O! Más newness!
Puns, for example, make new meaning by breaking established expectations - using clever turns-of-phrase - with a knack for impeccable timing. But not everybody enjoys hearing pun-smithery.
Perhaps inherent to the nature of punning is the punner's showmanship. By one-upping the listener's expectations time and time again, punning often turns away even the most pun-fun.


CODEX PHYLOGONIX
Philip K. Dick's etymolastical coinage, "phylogon," was first published in his 1981 sci-fi novel, "The Divine Invasion." The neologism describes the branching, netlike frameworks that accrete outside of space and time, but which tangentio-morphologically affect aspects of life within the space/time continuum.
In addition to first appearing in "The Divine Invasion," this abstruse term appears in his Exegesis abundantly. Yet we still lack any official - PKD-approved - definition. What remains is pure description.
At first blush, PKD's term seems like the brusque bastardization of "phylogeny" - meaning "the evolutionary history of a kind of organism" - and the geometrical suffix, "-gon" - meaning "having (so many) angles."
Together, these lexemes might suggest that the word "phylogon" describes the "many-sided histories of an organism throughout its evolution."
But, to PKD it seems to have signified something more.
This wasn't theory; it was fact.
In the 12th century, Europeans said that properties of "likeness" arise from mankind's attempt to restore our universal language after the fall of the Tower of Babel.
In the 1970's, American information scientists demonstrated how information content can be measured from surprise value thresholds being met, unmet, or exceeded.
In theory, this was fact.
Definition vs. Description:
While this rough piecing-together of PKD's neologism resembles Historical Linguistics 101, etymology can only help initiate the Exegete. The rest happens by way of practiced pragmatism, or, praxis.
Really,
That's abstruse.
If you Google "phylogon" today, you will discover that the term's use and utility remain with PKD's avid readers and expositors.
Plato called this type of mental image an "archetype." And Carl Jung emulated Plato's archetypes by applying their fundamental meaning to psychology. Many others have since realized that there are striking similarities between individual minds and our minds over time. But, what's that mean? And, what's an archetype doing in my mind anyway?

Keywords:
Apperception
Ap`per*cep"tion (#), n. Pref. ad- + perception: cf. F. apperception. The mind's perception of itself as the subject or actor in its own states; perception that reflects upon itself; sometimes, intensified or energetic perception. Leibnitz. Reid.
Archetype
A recurrent symbol or motif in literature, art, or mythology
(In Jungian theory) a primitive mental image inherited from the earliest human ancestors, and supposed to be present in the collective unconscious.
Morphology
+
Isomorphism
+
Reticulated
+
Arborized
+

